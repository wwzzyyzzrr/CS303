\documentclass[conference,compsoc]{IEEEtran}

\ifCLASSOPTIONcompsoc
 
  \usepackage[nocompress]{cite}
\else

  \usepackage{cite}
\fi

\usepackage{listings}
\usepackage{amsmath}
\usepackage{algorithm}
\usepackage{algorithmic}
\usepackage{algpseudocode}
\floatname{algorithm}{algorithm} 
\renewcommand{\algorithmicrequire}{\textbf{Input:}}
\renewcommand{\algorithmicensure}{\textbf{Output:}}

\usepackage{array}

\usepackage{url}

\hyphenation{op-tical net-works semi-conduc-tor}


\begin{document}

\title{Gomoku}



\author{\IEEEauthorblockN{Wang Zhiyuan  11610634}
\IEEEauthorblockA{CSE\\
Computer Science and Technology\\
11610634@mail.sustc.edu.cn}}

\maketitle
\IEEEpeerreviewmaketitle



\section{Preliminaries}

\subsection{Software}
For this project, I write it by Python. The package I have used is use $numpy$ and $copy$
\subsection{Algorithm}
For this Project, I use the method of heurlstlc search. The primary part of this Algorithm is the design of the 
evaluation function.  

And to optimize the Algorithm, I use the Min-Max Analysis to design a game tree, and use Alpha-Beta purning to simplify the process of search. 
But limited by the time. The depth of the tree should less then 8.

\section{Methodology}
\subsection{Representation}
In my code, according to the example given by the teacher, I design six method:  
\begin{itemize}
  \item $count()$
  \item $calcute\_value()$
  \item $get\_pos\_value()$
  \item $get\_pos\_list()$
  \item $tree()$
  \item $go()$
\end{itemize}
For these methods:
\begin{itemize}
  \item The $go()$ is the method that test program will call.
  \item The $count()$, $calcute\_value()$ and $get\_pos\_value()$ can calucute the value of each coordinate in the chessboard which is null now.
  \item The $get\_pos\_list()$ and $tree()$ will build a game tree.
\end{itemize}

\subsection{Architecture}
\begin{itemize}
  \item $go()$
  \begin{itemize}
    \item $count()$
    \item $calucute\_value()$
    \item $get\_pos\_value()$
    \item $calcute\_pos\_list()$
    \item $tree()$
    \begin{itemize}
      \item $count()$
      \item $calucute\_pos\_value()$
      \item $get\_pos\_value()$ 
      \item $calcute\_pos\_list()$
    \end{itemize}
  \end{itemize}
\end{itemize}


\subsection{Detail of Algorithm}
  Firstly, I need to design a evoluation function to get the value of all the coordinate with null color.
  I calculate the value of one location by combinate the conditions of 8 directions of this coordinate:
  \begin{itemize}
    \item Count how many chess with he same color as yours in one direction
    \item Count how many chess with the same color as yours if there is one null chess in one direction
    \item In the end of this direction is null chess or the versus color chess.
  \end{itemize}
  \begin{algorithm}
    \begin{algorithmic}[1]
    \end{algorithmic}
  \end{algorithm}

  After get the conditions of all the 8 directions of the null chess coordinates.
  I can combinate two direction in one line and get the result in this line.

  According to the conditions I get on the 4 lines, I can give weight value to this coordinate.

  


\section{Empirical Verification}


\subsection{Design}

\subsection{Data and data structure}

\subsection{Performance}

\subsection{Result}

\subsection{Analysis}



% use section* for acknowledgment
\ifCLASSOPTIONcompsoc
  % The Computer Society usually uses the plural form
  \section*{Acknowledgments}
\else
  % regular IEEE prefers the singular form
  \section*{Acknowledgment}
\fi







% trigger a \newpage just before the given reference
% number - used to balance the columns on the last page
% adjust value as needed - may need to be readjusted if
% the document is modified later
%\IEEEtriggeratref{8}
% The "triggered" command can be changed if desired:
%\IEEEtriggercmd{\enlargethispage{-5in}}

% references section

% can use a bibliography generated by BibTeX as a .bbl file
% BibTeX documentation can be easily obtained at:
% http://mirror.ctan.org/biblio/bibtex/contrib/doc/
% The IEEEtran BibTeX style support page is at:
% http://www.michaelshell.org/tex/ieeetran/bibtex/
\bibliographystyle{IEEEtran}
% argument is your BibTeX string definitions and bibliography database(s)
%\bibliography{IEEEabrv,../bib/paper}
%
% <OR> manually copy in the resultant .bbl file
% set second argument of \begin to the number of references
% (used to reserve space for the reference number labels box)
\begin{thebibliography}{1}
\bibitem{reference} XXXXXXX
\end{thebibliography}





% that's all folks
\end{document}


